\documentclass[11pt,a4paper,fleqn]{article}

\usepackage[ansinew]{inputenc}
\usepackage[mathscr]{eucal}
\usepackage{amsmath,amssymb,amsthm}
\usepackage{graphicx}
\usepackage{caption}
\usepackage{subcaption}
\usepackage{hyperref}

\allowdisplaybreaks
\flushbottom

\setlength{\textwidth}{160.0mm}
\setlength{\textheight}{245.0mm}
\setlength{\oddsidemargin}{0mm}
\setlength{\evensidemargin}{0mm}
\setlength{\topmargin}{-15mm} %{-20mm} for arXiv, {-15mm} for printing on A4
\setlength{\parindent}{5.0mm}

\hypersetup{colorlinks, linkcolor=blue, citecolor=blue, urlcolor=blue}

%\bibliographystyle{biblio}

\makeatletter
\long\def\@makecaption#1#2{%
  \vskip\abovecaptionskip\footnotesize
  \sbox\@tempboxa{#1. #2}%
  \ifdim \wd\@tempboxa >\hsize
    #1. #2\par
  \else
    \global \@minipagefalse
    \hb@xt@\hsize{\hfil\box\@tempboxa\hfil}%
  \fi
  \vskip\belowcaptionskip}
\makeatother

\marginparwidth=17mm \marginparsep=1mm \marginparpush=4mm

\newtheorem{theorem}{Theorem}
\newtheorem{lemma}{Lemma}
\newtheorem{corollary}{Corollary}
\newtheorem{proposition}{Proposition}
{\theoremstyle{definition}
\newtheorem{definition}{Definition}
\newtheorem{example}{Example}
\newtheorem{remark}{Remark}
\newtheorem*{remark*}{Remark}
}


\begin{document}

\par\noindent {\LARGE\bf
Modeling Forest Fires with Cellular Automata
\par}

\vspace{6mm}\par\noindent{\bf
Mark Power$^{\dag}$
}

\vspace{6mm}\par\noindent{\it
$^\dag$\,Department of Mathematics and Statistics, Memorial University of Newfoundland,\\
$\phantom{^\dag}$\,St.\ John's (NL) A1C 5S7, Canada
}

\vspace{6mm}\par\noindent
E-mails:
mkp302@mun.ca


\vspace{9mm}\par\noindent\hspace*{8mm}\parbox{140mm}{\small\looseness=-1
By using the Kingston fire as a case study, this paper aims to evaluate the effectiveness of a cellualar automata model in simulating the spread of the Kingston fire.. 
}\par\vspace{5mm}


Keywords:
Forest fire modelling,
Cellular automata,
Climate Data


\section{Introduction}
\subsection{Cellular Automata}
Cellular Autamata is a simple computational model used to simulate complex processes. It consists of an N by N grid of cells, with each cell having its own internal state. The state of each cell is updated by comparing its state to a neighbourhood of cells around it. The rules governing the update of cells are usually quite simple making the simulations less computationally demanding compared to physical models.

There are many practical applications of Cellular Automata simulations. One such application is the modeling of forest fires. In such a model, the cells of the grid represent sections of forest. This is a suitable model since a fire will typically spread locally.


\section{Case Study}
On August 3rd, 2025 a forest fire broke out near the town of Kingston, Newfoundland. 

\section{Data and Methods}\label{sec:Data and Methods}

\subsection{Datasets}
A number of datasets is used in the fire spread algorithm.

From the Eurpean Centre ER5 analysis dataset wind velocity (u,v) and temperature data was collected for the month of August. The data was 

From Opentopology, elevation data was collected from the NASA Shuttle Radar Topography Mission (SRTM). The 

\subsection{Algorithm}
In the Forest fire Model each cell can be in one of four states:

\begin{table}[h!]
\centering
\caption{Internal State of Cells}
\begin{tabular}{|l|l|p{6cm}|}
\hline
\textbf{State} & \textbf{Condition} & \textbf{Description} \\
\hline
0 & Non-flammable & 
The section of forest represents terrain features that are not flammable such as lakes or rivers \\
\hline
1:& Unburnt &
This section of forest contains fuel that has not yet ignited \\
\hline
2:& Burning &
This section of forest contains fuel and it is now burning \\
\hline
3:& Burnt out &
This section of forest has run out of fuel and is no longer burning\\
\hline


\end{tabular}
\end{table}

\begin{table}[h!]
\centering
\caption{Spread rate formulas for different slope and wind directions.}
\begin{tabular}{|l|l|}
\hline
\textbf{Condition} & \textbf{Spread Rate Formula} \\
\hline
Uphill &
$R_{\text{uphill}} =
0.969\, R_0\, K_s\, e^{3.533(\tan \phi)^{1.2} + 0.182\, V \cos \theta}$ \\[6pt]
\hline
Downhill &
$R_{\text{downhill}} =
0.969\, R_0\, K_s\, e^{-3.533(\tan \phi)^{1.2} + 0.182\, V \cos(180^\circ - \theta)}$ \\[6pt]
\hline
Left (flat slope) &
$R_{\text{left}} =
0.969\, R_0\, K_s\, e^{0.182\, V \cos(\theta + 90^\circ)}$ \\[6pt]
\hline
Right (flat slope) &
$R_{\text{right}} =
0.969\, R_0\, K_s\, e^{0.182\, V \cos(\theta - 90^\circ)}$ \\[6pt]
\hline
Wind (favorable direction) &
$R_{\text{wind}} =
0.969\, R_0\, K_s\, e^{3.533(\tan(\phi \cos \theta))^{1.2} + 0.182\, V},
\quad
\text{if } \theta \in [0^\circ, 90^\circ] \cup [270^\circ, 360^\circ]$ \\[6pt]
\hline
Wind (opposing direction) &
$R_{\text{wind}} =
0.969\, R_0\, K_s\, e^{-3.533(\tan(\phi \cos(180^\circ - \theta)))^{1.2} + 0.182\, V},
\quad
\text{if } 90^\circ < \theta < 270^\circ$ \\[6pt]
\hline
\end{tabular}
\end{table}

\begin{table}[h!]
\centering
\caption{Neighbour positions relative to cell $(i, j)$ and favourable wind directions.}
\begin{tabular}{|c|c|c|}
\hline
\textbf{Neighbour} & \textbf{Grid Offset $(\Delta i, \Delta j)$} & \textbf{Favourable Wind Direction (°)} \\
\hline
Top-left     & (-1, -1) & 315° (Northwest) \\
Top          & (-1,  0) & 0° (North) \\
Top-right    & (-1, +1) & 45° (Northeast) \\
Left         & ( 0, -1) & 270° (West) \\
Right        & ( 0, +1) & 90° (East) \\
Bottom-left  & (+1, -1) & 225° (Southwest) \\
Bottom       & (+1,  0) & 180° (South) \\
Bottom-right & (+1, +1) & 135° (Southeast) \\
\hline
\end{tabular}
\end{table}

\subsection{Implementation}

\section{Results}\label{sec:Results}

\section{Limitations}

\section{Conclusion}\label{sec:Conclusion}

\section{github}

\footnotesize\setlength{\itemsep}{0ex}
\bibliography{biblio}

\end{document}

