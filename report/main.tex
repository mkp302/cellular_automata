\documentclass[11pt,a4paper,fleqn]{article}

\usepackage[ansinew]{inputenc}
\usepackage[mathscr]{eucal}
\usepackage{amsmath,amssymb,amsthm}
\usepackage{graphicx}
\usepackage{caption}
\usepackage{subcaption}
\usepackage{hyperref}

\allowdisplaybreaks
\flushbottom

\setlength{\textwidth}{160.0mm}
\setlength{\textheight}{245.0mm}
\setlength{\oddsidemargin}{0mm}
\setlength{\evensidemargin}{0mm}
\setlength{\topmargin}{-15mm} %{-20mm} for arXiv, {-15mm} for printing on A4
\setlength{\parindent}{5.0mm}

\hypersetup{colorlinks, linkcolor=blue, citecolor=blue, urlcolor=blue}

%\bibliographystyle{biblio}

\makeatletter
\long\def\@makecaption#1#2{%
  \vskip\abovecaptionskip\footnotesize
  \sbox\@tempboxa{#1. #2}%
  \ifdim \wd\@tempboxa >\hsize
    #1. #2\par
  \else
    \global \@minipagefalse
    \hb@xt@\hsize{\hfil\box\@tempboxa\hfil}%
  \fi
  \vskip\belowcaptionskip}
\makeatother

\marginparwidth=17mm \marginparsep=1mm \marginparpush=4mm

\newtheorem{theorem}{Theorem}
\newtheorem{lemma}{Lemma}
\newtheorem{corollary}{Corollary}
\newtheorem{proposition}{Proposition}
{\theoremstyle{definition}
\newtheorem{definition}{Definition}
\newtheorem{example}{Example}
\newtheorem{remark}{Remark}
\newtheorem*{remark*}{Remark}
}


\begin{document}

\par\noindent {\LARGE\bf
Modeling Forest Fires with Cellular Automata
\par}

\vspace{6mm}\par\noindent{\bf
Mark Power$^{\dag}$
}

\vspace{6mm}\par\noindent{\it
$^\dag$\,Department of Mathematics and Statistics, Memorial University of Newfoundland,\\
$\phantom{^\dag}$\,St.\ John's (NL) A1C 5S7, Canada
}

\vspace{6mm}\par\noindent
E-mails:
\documentclass[11pt,a4paper,fleqn]{article}

\usepackage[ansinew]{inputenc}
\usepackage[mathscr]{eucal}
\usepackage{amsmath,amssymb,amsthm}
\usepackage{graphicx}
\usepackage{caption}
\usepackage{subcaption}
\usepackage{hyperref}

\allowdisplaybreaks
\flushbottom

\setlength{\textwidth}{160.0mm}
\setlength{\textheight}{245.0mm}
\setlength{\oddsidemargin}{0mm}
\setlength{\evensidemargin}{0mm}
\setlength{\topmargin}{-15mm} %{-20mm} for arXiv, {-15mm} for printing on A4
\setlength{\parindent}{5.0mm}

\hypersetup{colorlinks, linkcolor=blue, citecolor=blue, urlcolor=blue}

%\bibliographystyle{biblio}

\makeatletter
\long\def\@makecaption#1#2{%
  \vskip\abovecaptionskip\footnotesize
  \sbox\@tempboxa{#1. #2}%
  \ifdim \wd\@tempboxa >\hsize
    #1. #2\par
  \else
    \global \@minipagefalse
    \hb@xt@\hsize{\hfil\box\@tempboxa\hfil}%
  \fi
  \vskip\belowcaptionskip}
\makeatother

\marginparwidth=17mm \marginparsep=1mm \marginparpush=4mm

\newtheorem{theorem}{Theorem}
\newtheorem{lemma}{Lemma}
\newtheorem{corollary}{Corollary}
\newtheorem{proposition}{Proposition}
{\theoremstyle{definition}
\newtheorem{definition}{Definition}
\newtheorem{example}{Example}
\newtheorem{remark}{Remark}
\newtheorem*{remark*}{Remark}
}


\begin{document}

\par\noindent {\LARGE\bf
Modeling Forest Fires with Cellular Automata
\par}

\vspace{6mm}\par\noindent{\bf
Mark Power$^{\dag}$
}

\vspace{6mm}\par\noindent{\it
$^\dag$\,Department of Mathematics and Statistics, Memorial University of Newfoundland,\\
$\phantom{^\dag}$\,St.\ John's (NL) A1C 5S7, Canada
}

\vspace{6mm}\par\noindent
E-mails:
mkp302@mun.ca


\vspace{9mm}\par\noindent\hspace*{8mm}\parbox{140mm}{\small\looseness=-1
By using the Kingston fire as a case study, this paper aims to evaluate the effectiveness of a cellualar automata model in simulating the spread of the Kingston fire.. 
}\par\vspace{5mm}


Keywords:
Forest fire modelling,
Cellular automata,
Climate Data


\section{Introduction}
Cellular Autamata is a simple computational model used to simulate complex processes. It consists of an N by N grid of cells, with each cell having its own internal state. The state of each cell is updated by comparing its state to a neighbourhood of cells around it.

The rules of 

\section{Case Study}
On August 3rd, 2025 a forest fire broke out near the town of Kingston, Newfoundland. 

\section{Data and Methods}\label{sec:Data and Methods}

\subsection{Datasets}

\subsection{Algorithm}

\subsection{Implementation}

\section{Results}\label{sec:Results}

\section{Conclusion}\label{sec:Conclusion}

\section{github}

\footnotesize\setlength{\itemsep}{0ex}
\bibliography{biblio}

\end{document}
:\While{}
  \State 
\EndWhile
