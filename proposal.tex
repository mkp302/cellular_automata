\documentclass{article}
\usepackage{graphicx} % Required for inserting images

\title{Research Proposal}
\author{Mark Power}
\date{October 2025}
\documentclass{article}
\usepackage{graphicx} % Required for inserting images

\title{Research Proposal}
\author{Mark Power}
\date{October 2025}

\begin{document}

\maketitle

\section{Introduction}
The goal of this research is to evaluate the use of cellular automata in modelling the forest fires that occurred in Newfoundland during the Summer of 2025.

Cellular Automaton consist of a grid of cells and use simple rules to model complex processes. Useful for simulations where each successive state only depends on the state of its immediate neighbours. When considering which cells to use when updating two popular methods are Von Neuman and Moore Neighbourhoods.

One application of Cellular Automaton is to model the spread of forest fires.

A Model for how fire spreads is the Wang Zhengfei. This models considers fuel type, wind speed, and slope as parameters driving how a fire spreads

$$R = R_0 K_w K_s / cos \alpha$$
where $R_0 is the initial fire spread rate, $K_w$ is the wind velocity correction factor, $K_s$ is the fuel correction factor. $\alpha$is the average terrain slope.

\section{Methodology}
    \subsection{Data Acquisition}
        \begin{list}
        \end{list}

    \subsection{Algorithm}
    
    \subsection{Evaluation}
    
    
 \section{References}
 https://c-core.ca/solutions/newfoundland-wildfires-as-seen-from-space/
\newline
 https://cds.climate.copernicus.eu/datasets/reanalysis-era5-single-levels-timeseries?tab=download
 \section{github}
 https://github.com/mkp302/cellular\_automata
 

\end{document}

